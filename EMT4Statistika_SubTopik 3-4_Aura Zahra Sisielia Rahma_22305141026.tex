\documentclass{article}

\usepackage{eumat}

\begin{document}
\begin{eulernotebook}
\begin{eulercomment}
Nama  : Aura Zahra Sisielia Rahma\\
NIM   : 22305141026\\
Kelas : Matematika B

\begin{eulercomment}
\eulerheading{File Input dan Output (Membaca dan Menulis Data) }
\begin{eulercomment}
Kali ini kita akan belajar bagaimana cara mengimpor matriks data dari
sumber lain ke EMT. Fungsi sederhana adalah writematrix() dan
readmatrix().

-Fungsi writematrix() digunakan untuk menulis matriks ke dalam file
teks. Ini memungkinkan Anda untuk menyimpan data matriks dalam format
yang dapat dibaca oleh program lain atau untuk tujuan penyimpanan.

-Fungsi readmatrix() digunakan untuk membaca data matriks dari file
teks. Ini memungkinkan Anda untuk mengambil data matriks yang telah
disimpan sebelumnya dalam file teks.


Ditunjukan cara membaca dan menulis vektor real ke dalam file.
\end{eulercomment}
\begin{eulerprompt}
>a=random(1,100)
\end{eulerprompt}
\begin{euleroutput}
  [0.655416,  0.200995,  0.893622,  0.281887,  0.525,  0.314127,
  0.444616,  0.299474,  0.28269,  0.883227,  0.270906,  0.704419,
  0.217693,  0.445363,  0.308411,  0.914541,  0.193585,  0.463387,
  0.095153,  0.595017,  0.431184,  0.72868,  0.465164,  0.323032,
  0.525184,  0.502255,  0.168603,  0.262253,  0.866587,  0.536137,
  0.493453,  0.601344,  0.659461,  0.967468,  0.193151,  0.935921,
  0.0728753,  0.988966,  0.0104376,  0.356626,  0.52143,  0.428893,
  0.168134,  0.182742,  0.288048,  0.750042,  0.472935,  0.324407,
  0.340388,  0.195494,  0.44607,  0.216545,  0.598477,  0.729271,
  0.950019,  0.791698,  0.418637,  0.469159,  0.898763,  0.388543,
  0.674114,  0.0752676,  0.854693,  0.123222,  0.205006,  0.465049,
  0.0281269,  0.60809,  0.10999,  0.615836,  0.217391,  0.962872,
  0.842267,  0.328758,  0.561146,  0.918802,  0.997492,  0.473212,
  0.185709,  0.421585,  0.720843,  0.247804,  0.834083,  0.44372,
  0.371866,  0.775718,  0.37468,  0.224336,  0.0102885,  0.164816,
  0.356866,  0.408866,  0.350091,  0.347412,  0.161257,  0.588325,
  0.83884,  0.906718,  0.916402,  0.0960726]
\end{euleroutput}
\begin{eulerprompt}
>mean(a), dev(a)
\end{eulerprompt}
\begin{euleroutput}
  0.474656416958
  0.273269044817
\end{euleroutput}
\begin{eulercomment}
Seperti yang sudah kita ketahui mean() disini berarti rata-rata.\\
Rumus mean (rata-rata) adalah

\end{eulercomment}
\begin{eulerformula}
\[
Mean = \frac{sum(data)}{len(data)}
\]
\end{eulerformula}
\begin{eulercomment}
atau jumlah data dibagi banyak data.

Sedangkan dev() merupakan standartdeviasinya.\\
Rumus standart deviasi adalah

\end{eulercomment}
\begin{eulerformula}
\[
deviasi= [(x-mean)^{2} for x in data]
\]
\end{eulerformula}
\begin{eulerformula}
\[
variasi= \frac{sum(deviasi)}{len(data)}
\]
\end{eulerformula}
\begin{eulerformula}
\[
standard deviasi=\sqrt{variasi}
\]
\end{eulerformula}
\begin{eulerprompt}
>filename="test.dat";
\end{eulerprompt}
\begin{eulercomment}
Sekarang kita menulis vektor kolom a' ke file. Ini menghasilkan satu
nomor di setiap baris file.
\end{eulercomment}
\begin{eulerprompt}
>writematrix(a',filename);
\end{eulerprompt}
\begin{eulercomment}
Untuk membaca data, kami menggunakan readmatrix().
\end{eulercomment}
\begin{eulerprompt}
>a=readmatrix(filename)';
\end{eulerprompt}
\begin{eulercomment}
Dan hapus filenya.
\end{eulercomment}
\begin{eulerprompt}
>fileremove(filename);
>mean(a), dev(a),
\end{eulerprompt}
\begin{euleroutput}
  0.474656416958
  0.273269044817
\end{euleroutput}
\begin{eulercomment}
Contoh lain
\end{eulercomment}
\begin{eulerprompt}
>b= random(1,10)
\end{eulerprompt}
\begin{euleroutput}
  [0.978964,  0.0367821,  0.191109,  0.642479,  0.233536,  0.0910807,
  0.899334,  0.983636,  0.21102,  0.552777]
\end{euleroutput}
\begin{eulerprompt}
>mean(b), dev(b),
\end{eulerprompt}
\begin{euleroutput}
  0.482071750833
  0.376292059882
\end{euleroutput}
\begin{eulerprompt}
>writematrix(b',filename);
>b=readmatrix(filename)';
>fileremove(filename);
>mean(b), dev(b),
\end{eulerprompt}
\begin{euleroutput}
  0.482071750833
  0.376292059882
\end{euleroutput}
\begin{eulerprompt}
>c=[2,4,6]; mean(c), dev(c), median(c)
\end{eulerprompt}
\begin{euleroutput}
  4
  2
  4
\end{euleroutput}
\begin{eulercomment}
Penjelasan:

Mencari rata-ratanya\\
\end{eulercomment}
\begin{eulerformula}
\[
Rata-rata= \frac{2+4+6}{3}
\]
\end{eulerformula}
\begin{eulerformula}
\[
= \frac{12}{3}
\]
\end{eulerformula}
\begin{eulerformula}
\[
= 4
\]
\end{eulerformula}
\begin{eulercomment}
Lalu mencari standart deviasinya

-Deviasi\\
\end{eulercomment}
\begin{eulerformula}
\[
(2-4)^{2} = 4
\]
\end{eulerformula}
\begin{eulerformula}
\[
(4-4)^{2} = 0
\]
\end{eulerformula}
\begin{eulerformula}
\[
(6-4)^{2} = 4
\]
\end{eulerformula}
\begin{eulercomment}
-Variasi\\
\end{eulercomment}
\begin{eulerformula}
\[
variasi = \frac{4+0+4}{3}
\]
\end{eulerformula}
\begin{eulerformula}
\[
= 2.6
\]
\end{eulerformula}
\begin{eulercomment}
-Standart deviasi\\
\end{eulercomment}
\begin{eulerformula}
\[
standar deviasi= \sqrt{2.6}
\]
\end{eulerformula}
\begin{eulerformula}
\[
= 1.632
\]
\end{eulerformula}
\begin{eulercomment}
dibulatkan\\
\end{eulercomment}
\begin{eulerformula}
\[
= 2
\]
\end{eulerformula}
\begin{eulerprompt}
> 
>d=random(100)
\end{eulerprompt}
\begin{euleroutput}
  [0.494142,  0.579784,  0.811875,  0.983877,  0.262956,  0.993804,
  0.901832,  0.240796,  0.635967,  0.151086,  0.251504,  0.333971,
  0.761139,  0.410119,  0.5587,  0.711308,  0.0843473,  0.320241,
  0.353311,  0.433255,  0.424885,  0.022074,  0.838671,  0.42851,
  0.135752,  0.210168,  0.793448,  0.636726,  0.264161,  0.51778,
  0.435232,  0.421074,  0.670718,  0.827047,  0.762061,  0.473647,
  0.770606,  0.590967,  0.0638119,  0.165065,  0.796324,  0.718715,
  0.486728,  0.270117,  0.430124,  0.172917,  0.0976369,  0.99418,
  0.00359736,  0.647971,  0.52536,  0.933253,  0.748021,  0.763355,
  0.571752,  0.275951,  0.280553,  0.0664734,  0.221311,  0.0372897,
  0.450195,  0.487021,  0.521941,  0.703232,  0.384655,  0.679077,
  0.124481,  0.489252,  0.361611,  0.0303537,  0.572118,  0.245745,
  0.197327,  0.265461,  0.791105,  0.276179,  0.908966,  0.641266,
  0.602455,  0.704627,  0.0212425,  0.465424,  0.181763,  0.142573,
  0.206751,  0.853878,  0.944496,  0.142014,  0.213859,  0.818767,
  0.532792,  0.153921,  0.450978,  0.844721,  0.550462,  0.333096,
  0.434177,  0.904402,  0.354328,  0.373107]
\end{euleroutput}
\begin{eulerprompt}
>filename="aura.dat";
>d
\end{eulerprompt}
\begin{euleroutput}
  [0.494142,  0.579784,  0.811875,  0.983877,  0.262956,  0.993804,
  0.901832,  0.240796,  0.635967,  0.151086,  0.251504,  0.333971,
  0.761139,  0.410119,  0.5587,  0.711308,  0.0843473,  0.320241,
  0.353311,  0.433255,  0.424885,  0.022074,  0.838671,  0.42851,
  0.135752,  0.210168,  0.793448,  0.636726,  0.264161,  0.51778,
  0.435232,  0.421074,  0.670718,  0.827047,  0.762061,  0.473647,
  0.770606,  0.590967,  0.0638119,  0.165065,  0.796324,  0.718715,
  0.486728,  0.270117,  0.430124,  0.172917,  0.0976369,  0.99418,
  0.00359736,  0.647971,  0.52536,  0.933253,  0.748021,  0.763355,
  0.571752,  0.275951,  0.280553,  0.0664734,  0.221311,  0.0372897,
  0.450195,  0.487021,  0.521941,  0.703232,  0.384655,  0.679077,
  0.124481,  0.489252,  0.361611,  0.0303537,  0.572118,  0.245745,
  0.197327,  0.265461,  0.791105,  0.276179,  0.908966,  0.641266,
  0.602455,  0.704627,  0.0212425,  0.465424,  0.181763,  0.142573,
  0.206751,  0.853878,  0.944496,  0.142014,  0.213859,  0.818767,
  0.532792,  0.153921,  0.450978,  0.844721,  0.550462,  0.333096,
  0.434177,  0.904402,  0.354328,  0.373107]
\end{euleroutput}
\begin{eulerprompt}
>mean(d), dev(d), median(d)
\end{eulerprompt}
\begin{euleroutput}
  0.471278663938
  0.2714826998
  0.450586280925
\end{euleroutput}
\begin{eulercomment}
Fungsi writematrix() atau writetable() dapat dikonfigurasi untuk
bahasa lain.

Misalnya, jika Anda memiliki sistem Indonesia (titik desimal dengan
koma), Excel Anda memerlukan nilai dengan koma desimal yang dipisahkan
oleh titik koma dalam file csv (defaultnya adalah nilai yang
dipisahkan koma). File "test.csv" berikut akan muncul di folder
cuurent Anda.
\end{eulercomment}
\begin{eulerprompt}
>filename="text.csv"; ...
>writematrix(random(5,3),file=filename,separator=",");
\end{eulerprompt}
\begin{eulercomment}
Anda sekarang dapat membuka file ini dengan Excel Indonesia secara
langsung.
\end{eulercomment}
\begin{eulerprompt}
>fileremove(filename);
\end{eulerprompt}
\begin{eulercomment}
Terkadang kita memiliki string dengan token seperti berikut ini.
\end{eulercomment}
\begin{eulerprompt}
>s1:="f m m f m m m f f f m m f"; ...
>s2:="f f f m m f f";
\end{eulerprompt}
\begin{eulercomment}
Untuk tokenize ini, kami mendefinisikan vektor token.
\end{eulercomment}
\begin{eulerprompt}
>tok:=["f","m"]
\end{eulerprompt}
\begin{euleroutput}
  f
  m
\end{euleroutput}
\begin{eulercomment}
Kemudian kita dapat menghitung berapa kali setiap token muncul dalam
string, dan memasukkan hasilnya ke dalam tabel.
\end{eulercomment}
\begin{eulerprompt}
> M:=getmultiplicities(tok,strtokens(s1))_ ...
>getmultiplicities(tok,strtokens(s2));
\end{eulerprompt}
\begin{eulercomment}
Tulis tabel dengan header token.
\end{eulercomment}
\begin{eulerprompt}
>writetable(M,labc=tok,labr=1:2,wc=6)
\end{eulerprompt}
\begin{euleroutput}
             f     m
       1     6     7
       2     5     2
\end{euleroutput}
\begin{eulercomment}
Untuk statistika, EMT dapat membaca dan menulis tabel.
\end{eulercomment}
\begin{eulerprompt}
>file="test.dat"; open(file,"w"); ...
>writeln("A,B,C"); writematrix(random(3,3)); ...
>close();
\end{eulerprompt}
\begin{eulercomment}
Filenya akan terlihat seperti ini.
\end{eulercomment}
\begin{eulerprompt}
>printfile(file)
\end{eulerprompt}
\begin{euleroutput}
  A,B,C
  0.1487056648544881,0.0006365920668981623,0.5149201954566951
  0.7031180876144383,0.5381325719604588,0.7812141493909074
  0.9499312870875743,0.4223447897933922,0.6092230024672836
  
\end{euleroutput}
\begin{eulercomment}
Fungsi readtable() dalam bentuknya yang paling sederhana dapat membaca
ini dan mengembalikan kumpulan nilai dan baris judul.
\end{eulercomment}
\begin{eulerprompt}
>L=readtable(file,>list);
\end{eulerprompt}
\begin{eulercomment}
Koleksi ini dapat dicetak dengan writetable() ke notebook, atau ke
file.
\end{eulercomment}
\begin{eulerprompt}
>writetable(L,wc=10 ,dc=5)
\end{eulerprompt}
\begin{euleroutput}
           A         B         C
     0.14871   0.00064   0.51492
     0.70312   0.53813   0.78121
     0.94993   0.42234   0.60922
\end{euleroutput}
\begin{eulercomment}
Matriks nilai adalah elemen pertama dari L. Perhatikan bahwa mean()
dalam EMT menghitung nilai rata-rata dari baris matriks.
\end{eulercomment}
\begin{eulerprompt}
>mean(L[1])
\end{eulerprompt}
\begin{euleroutput}
       0.221421 
       0.674155 
         0.6605 
\end{euleroutput}
\eulerheading{File CSV}
\begin{eulercomment}
Pertama, mari kita menulis matriks ke dalam file. Untuk output, kami
membuat file di direktori kerja saat ini.
\end{eulercomment}
\begin{eulerprompt}
>file="test.csv"; ...
>M=random(4,4); writematrix(M,file);
\end{eulerprompt}
\begin{eulercomment}
Berikut adalah isi dari file ini.
\end{eulercomment}
\begin{eulerprompt}
>printfile(file)
\end{eulerprompt}
\begin{euleroutput}
  0.9539904492391218,0.3242501283815414,0.3509798221445106,0.4135727822564945
  0.8170264863629133,0.2966076923588541,0.8654591047968827,0.8838164965270098
  0.271055532172192,0.2311736562521014,0.3945122601154922,0.780445814755696
  0.5812301348916076,0.5930523557632464,0.2462355447607214,0.4554301820283595
  
\end{euleroutput}
\begin{eulercomment}
CVS ini dapat dibuka pada sistem bahasa Inggris ke Excel dengan klik
dua kali. Jika Anda mendapatkan file seperti itu di sistem Jerman,
Anda perlu mengimpor data ke Excel dengan memperhatikan titik desimal.

Tetapi titik desimal juga merupakan format default untuk EMT. Anda
dapat membaca matriks dari file dengan readmatrix().
\end{eulercomment}
\begin{eulerprompt}
>readmatrix(file)
\end{eulerprompt}
\begin{euleroutput}
        0.95399       0.32425       0.35098      0.413573 
       0.817026      0.296608      0.865459      0.883816 
       0.271056      0.231174      0.394512      0.780446 
        0.58123      0.593052      0.246236       0.45543 
\end{euleroutput}
\begin{eulercomment}
Dimungkinkan untuk menulis beberapa matriks ke satu file. Perintah
open() dapat membuka file untuk ditulis dengan parameter "w".
Standarnya adalah "r" untuk membaca.
\end{eulercomment}
\begin{eulerprompt}
>open(file,"w"); writematrix(M); writematrix(M'); close();
\end{eulerprompt}
\begin{eulercomment}
Matriks dipisahkan oleh garis kosong. Untuk membaca matriks, buka file\\
dan panggil readmatrix() beberapa kali.
\end{eulercomment}
\begin{eulerprompt}
>open(file); A=readmatrix(); B=readmatrix(); A==B, close();
\end{eulerprompt}
\begin{euleroutput}
              1             0             0             0 
              0             1             0             0 
              0             0             1             0 
              0             0             0             1 
\end{euleroutput}
\begin{eulercomment}
Di Excel atau spreadsheet serupa, Anda dapat mengekspor matriks
sebagai CSV (nilai yang dipisahkan koma). Di Excel 2007, gunakan
"simpan sebagai" dan "format lain", lalu pilih "CSV". Pastikan, tabel
saat ini hanya berisi data yang ingin Anda ekspor.

Berikut adalah contoh.
\end{eulercomment}
\begin{eulerprompt}
>printfile("excel-data.csv")
\end{eulerprompt}
\begin{euleroutput}
  0;1000;1000
  1;1051,271096;1072,508181
  2;1105,170918;1150,273799
  3;1161,834243;1233,67806
  4;1221,402758;1323,129812
  5;1284,025417;1419,067549
  6;1349,858808;1521,961556
  7;1419,067549;1632,31622
  8;1491,824698;1750,6725
  9;1568,312185;1877,610579
  10;1648,721271;2013,752707
\end{euleroutput}
\begin{eulercomment}
Seperti yang Anda lihat, sistem Jerman saya menggunakan titik koma
sebagai pemisah dan koma desimal. Anda dapat mengubah ini di
pengaturan sistem atau di Excel, tetapi tidak perlu membaca matriks ke
dalam EMT.

Cara termudah untuk membaca ini ke dalam Euler adalah readmatrix().
Semua koma diganti dengan titik dengan parameter \textgreater{}comma. Untuk CSV
bahasa Inggris, cukup abaikan parameter ini.
\end{eulercomment}
\begin{eulerprompt}
>M=readmatrix("excel-data.csv",>comma)
\end{eulerprompt}
\begin{euleroutput}
              0          1000          1000 
              1       1051.27       1072.51 
              2       1105.17       1150.27 
              3       1161.83       1233.68 
              4        1221.4       1323.13 
              5       1284.03       1419.07 
              6       1349.86       1521.96 
              7       1419.07       1632.32 
              8       1491.82       1750.67 
              9       1568.31       1877.61 
             10       1648.72       2013.75 
\end{euleroutput}
\begin{eulercomment}
Mari kita plot ini.
\end{eulercomment}
\begin{eulerprompt}
>plot2d(M'[1],M'[2:3],>points,color=[red,green]'):
\end{eulerprompt}
\eulerimg{27}{images/EMT4Statistika_SubTopik 3-4_Aura Zahra Sisielia Rahma_22305141026-005.png}
\begin{eulercomment}
Ada cara yang lebih mendasar untuk membaca data dari file. Anda dapat
membuka file dan membaca angka baris demi baris. Fungsi
getvectorline() akan membaca angka dari baris data. Secara default, ia
mengharapkan titik desimal. Tapi itu juga bisa menggunakan koma
desimal, jika Anda memanggil setdecimaldot(",") sebelum Anda
menggunakan fungsi ini.

Fungsi berikut adalah contoh untuk ini. Ini akan berhenti di akhir
file atau baris kosong.
\end{eulercomment}
\begin{eulerprompt}
>function myload (file) ...
\end{eulerprompt}
\begin{eulerudf}
  open(file);
  M=[];
  repeat
     until eof();
     v=getvectorline(3);
     if length(v)>0 then M=M_v; else break; endif;
  end;
  return M;
  close(file);
  endfunction
\end{eulerudf}
\begin{eulerprompt}
>myload(file)
\end{eulerprompt}
\begin{euleroutput}
        0.95399       0.32425       0.35098      0.413573 
       0.817026      0.296608      0.865459      0.883816 
       0.271056      0.231174      0.394512      0.780446 
        0.58123      0.593052      0.246236       0.45543 
\end{euleroutput}
\begin{eulercomment}
Dimungkinkan juga untuk membaca semua angka dalam file itu dengan
getvector().
\end{eulercomment}
\begin{eulerprompt}
> open(file); v=getvector(10000); close(); redim(v[1:9],3,3)
\end{eulerprompt}
\begin{euleroutput}
        0.95399       0.32425       0.35098 
       0.413573      0.817026      0.296608 
       0.865459      0.883816      0.271056 
\end{euleroutput}
\begin{eulercomment}
Jadi sangat mudah untuk menyimpan vektor nilai, satu nilai di setiap
baris dan membaca kembali vektor ini.
\end{eulercomment}
\begin{eulerprompt}
>v=random(1000); mean(v)
\end{eulerprompt}
\begin{euleroutput}
  0.510431026213
\end{euleroutput}
\begin{eulerprompt}
>writematrix(v',file); mean(readmatrix(file)')
\end{eulerprompt}
\begin{euleroutput}
  0.510431026213
\end{euleroutput}
\begin{eulercomment}
Contoh lain\\
Disini saya menggunakan file csv yang lain yaitu mtcars
\end{eulercomment}
\begin{eulerprompt}
>fileku="mtcars.csv"; ...
>N=random(3,3); writematrix(N,fileku);
>printfile(fileku)
\end{eulerprompt}
\begin{euleroutput}
  0.09719871136447902,0.4578649713278388,0.7048448053709271
  0.2429406296694435,0.1932052319973915,0.2449390659201055
  0.4284328615371106,0.2442040079074635,0.8627321782206349
  
\end{euleroutput}
\begin{eulerprompt}
>readmatrix(fileku)
\end{eulerprompt}
\begin{euleroutput}
      0.0971987      0.457865      0.704845 
       0.242941      0.193205      0.244939 
       0.428433      0.244204      0.862732 
\end{euleroutput}
\begin{eulerprompt}
>mean(N[1])
\end{eulerprompt}
\begin{euleroutput}
  0.419969496021
\end{euleroutput}
\eulerheading{Berlatih Membaca dan Menulis File}
\begin{eulercomment}
Menggunakan file yang disediakan dibesmart
\end{eulercomment}
\begin{eulerprompt}
>filename="table.dat";
>printfile(filename)
\end{eulerprompt}
\begin{euleroutput}
  Person Sex Age Titanic Evaluation Tip Problem
  1 m 30 n . 1.80 n
  2 f 23 y g 1.80 n
  3 f 26 y g 1.80 y
  4 m 33 n . 2.80 n
  5 m 37 n . 1.80 n
  6 m 28 y g 2.80 y
  7 f 31 y vg 2.80 n
  8 m 23 n . 0.80 n
  9 f 24 y vg 1.80 y
  10 m 26 n . 1.80 n
  11 f 23 y vg 1.80 y
  12 m 32 y g 1.80 n
  13 m 29 y vg 1.80 y
  14 f 25 y g 1.80 y
  15 f 31 y g 0.80 n
  16 m 26 y g 2.80 n
  17 m 37 n . 3.80 n
  18 m 38 y g . n
  19 f 29 n . 3.80 n
  20 f 28 y vg 1.80 n
  21 f 28 y m 2.80 y
  22 f 28 y vg 1.80 y
  23 f 38 y g 2.80 n
  24 f 27 y m 1.80 y
  25 m 27 n . 2.80 y
\end{euleroutput}
\begin{eulercomment}
2. 
\end{eulercomment}
\begin{eulerprompt}
>filename="table1.dat";
>printfile(filename,6)
\end{eulerprompt}
\begin{euleroutput}
  Person Sex Age Mother Father Siblings
  1 m 29 58 61 1
  2 f 26 53 54 2
  3 m 24 49 55 1
  4 f 25 56 63 3
  5 f 25 49 53 0
\end{euleroutput}
\begin{eulercomment}
3.
\end{eulercomment}
\begin{eulerprompt}
>fileku="GOOG.csv";
>printfile(fileku)
\end{eulerprompt}
\begin{euleroutput}
  Date,Open,High,Low,Close,Adj Close,Volume
  2019-11-04,1276.449951,1323.739990,1276.354980,1311.369995,1311.369995,7217800
  2019-11-11,1303.180054,1334.880005,1293.510010,1334.869995,1334.869995,5900600
  2019-11-18,1332.219971,1335.529053,1291.150024,1295.339966,1295.339966,6446400
  2019-11-25,1299.180054,1318.359985,1298.130005,1304.959961,1304.959961,3688500
  2019-12-02,1301.000000,1344.000000,1279.000000,1340.619995,1340.619995,6719700
  2019-12-09,1338.040039,1359.449951,1336.040039,1347.829956,1347.829956,6129400
  2019-12-16,1356.500000,1365.000000,1348.984985,1349.589966,1349.589966,9558800
  2019-12-23,1355.869995,1364.530029,1342.780029,1351.890015,1351.890015,2936500
  2019-12-30,1350.000000,1372.500000,1329.084961,1360.660034,1360.660034,4605700
  2020-01-06,1350.000000,1434.928955,1350.000000,1429.729980,1429.729980,8084600
  2020-01-13,1436.130005,1481.295044,1426.020020,1480.390015,1480.390015,8063800
  2020-01-20,1479.119995,1503.213989,1465.250000,1466.709961,1466.709961,6783300
  2020-01-27,1431.000000,1470.130005,1421.199951,1434.229980,1434.229980,8166900
  2020-02-03,1462.000000,1490.000000,1426.300049,1479.229980,1479.229980,11826100
  2020-02-10,1474.319946,1529.630005,1474.319946,1520.739990,1520.739990,6059400
  2020-02-17,1515.000000,1532.105957,1480.439941,1485.109985,1485.109985,4898300
  2020-02-24,1426.109985,1438.140015,1271.000000,1339.329956,1339.329956,14316700
  2020-03-02,1351.609985,1410.150024,1261.050049,1298.410034,1298.410034,11969000
  2020-03-09,1205.300049,1281.150024,1113.300049,1219.729980,1219.729980,16512100
  2020-03-16,1096.000000,1157.969971,1037.280029,1072.319946,1072.319946,19600200
  2020-03-23,1061.319946,1169.969971,1013.536011,1110.709961,1110.709961,18250300
  2020-03-30,1125.040039,1175.310059,1079.810059,1097.880005,1097.880005,11683000
  2020-04-06,1138.000000,1225.569946,1130.939941,1211.449951,1211.449951,9202500
  2020-04-13,1209.180054,1294.430054,1187.598022,1283.250000,1283.250000,10349000
  2020-04-20,1271.000000,1293.310059,1209.709961,1279.310059,1279.310059,9148200
  2020-04-27,1296.000000,1359.989990,1232.199951,1320.609985,1320.609985,13086900
  2020-05-04,1308.229980,1398.760010,1299.000000,1388.369995,1388.369995,7156600
  2020-05-11,1378.280029,1416.530029,1323.910034,1373.189941,1373.189941,7924600
  2020-05-18,1361.750000,1415.489990,1354.250000,1410.420044,1410.420044,7454400
  2020-05-25,1437.270020,1441.000000,1391.290039,1428.920044,1428.920044,7276700
  2020-06-01,1418.390015,1446.552002,1404.729980,1438.390015,1438.390015,6970600
  2020-06-08,1422.339966,1474.259033,1386.020020,1413.180054,1413.180054,8274100
  2020-06-15,1390.800049,1460.000000,1387.920044,1431.719971,1431.719971,9501200
  2020-06-22,1429.000000,1475.941040,1351.989990,1359.900024,1359.900024,10226400
  2020-06-29,1358.180054,1482.949951,1347.010010,1464.699951,1464.699951,7486900
  2020-07-06,1480.060059,1543.829956,1472.859985,1541.739990,1541.739990,7551500
  2020-07-13,1550.000000,1577.131958,1483.500000,1515.550049,1515.550049,8018100
  2020-07-20,1515.260010,1586.989990,1488.400024,1511.869995,1511.869995,6879500
  2020-07-27,1515.599976,1540.969971,1454.030029,1482.959961,1482.959961,9166000
  2020-08-03,1486.640015,1516.844971,1458.650024,1494.489990,1494.489990,9785200
  2020-08-10,1487.180054,1537.250000,1473.079956,1507.729980,1507.729980,6991400
  2020-08-17,1514.670044,1597.719971,1507.969971,1580.420044,1580.420044,8219400
  2020-08-24,1593.979980,1659.219971,1580.569946,1644.410034,1644.410034,11011800
  2020-08-31,1647.890015,1733.180054,1547.613037,1591.040039,1591.040039,11877700
  2020-09-07,1533.510010,1584.081055,1497.359985,1520.719971,1520.719971,7601300
  2020-09-14,1539.005005,1564.000000,1437.130005,1459.989990,1459.989990,9323100
  2020-09-21,1440.060059,1469.520020,1406.550049,1444.959961,1444.959961,8902600
  2020-09-28,1474.209961,1499.040039,1449.301025,1458.420044,1458.420044,7750300
  2020-10-05,1466.209961,1516.520020,1436.000000,1515.219971,1515.219971,6728000
  2020-10-12,1543.000000,1593.859985,1532.569946,1573.010010,1573.010010,8989300
  2020-10-19,1580.459961,1642.359985,1525.670044,1641.000000,1641.000000,9226500
  2020-10-26,1625.010010,1687.000000,1514.619995,1621.010010,1621.010010,11248500
  2020-11-02,null,null,null,null,null,null
  2020-11-02,1628.160034,1660.674927,1627.652466,1638.349976,1638.349976,1278888
\end{euleroutput}
\begin{eulercomment}
4
\end{eulercomment}
\begin{eulerprompt}
>filename="sample.csv";
>printfile(filename)
\end{eulerprompt}
\begin{euleroutput}
  female,read,write,math,hon,femalexmath
  0,57,52,41,0,0
  1,68,59,53,0,53
  0,44,33,54,0,0
  0,63,44,47,0,0
  0,47,52,57,0,0
  0,44,52,51,0,0
  0,50,59,42,0,0
  0,34,46,45,0,0
  0,63,57,54,0,0
  0,57,55,52,0,0
  0,60,46,51,0,0
  0,57,65,51,1,0
  0,73,60,71,0,0
  0,54,63,57,1,0
  0,45,57,50,0,0
  0,42,49,43,0,0
  0,47,52,51,0,0
  0,57,57,60,0,0
  0,68,65,62,1,0
  0,55,39,57,0,0
  0,63,49,35,0,0
  0,63,63,75,1,0
  0,50,40,45,0,0
  0,60,52,57,0,0
  0,37,44,45,0,0
  0,34,37,46,0,0
  0,65,65,66,1,0
  0,47,57,57,0,0
  0,44,38,49,0,0
  0,52,44,49,0,0
  0,42,31,57,0,0
  0,76,52,64,0,0
  0,65,67,63,1,0
  0,42,41,57,0,0
  0,52,59,50,0,0
  0,60,65,58,1,0
  0,68,54,75,0,0
  0,65,62,68,1,0
  0,47,31,44,0,0
  0,39,31,40,0,0
  0,47,47,41,0,0
  0,55,59,62,0,0
  0,52,54,57,0,0
  0,42,41,43,0,0
  0,65,65,48,1,0
  0,55,59,63,0,0
  0,50,40,39,0,0
  0,65,59,70,0,0
  0,47,59,63,0,0
  0,57,54,59,0,0
  0,53,61,61,1,0
  0,39,33,38,0,0
  0,44,44,61,0,0
  0,63,59,49,0,0
  0,73,62,73,1,0
  0,39,39,44,0,0
  0,37,37,42,0,0
  0,42,39,39,0,0
  0,63,57,55,0,0
  0,48,49,52,0,0
  0,50,46,45,0,0
  0,47,62,61,1,0
  0,44,44,39,0,0
  0,34,33,41,0,0
  0,50,42,50,0,0
  0,44,41,40,0,0
  0,60,54,60,0,0
  0,47,39,47,0,0
  0,63,43,59,0,0
  0,50,33,49,0,0
  0,44,44,46,0,0
  0,60,54,58,0,0
  0,73,67,71,1,0
  0,68,59,58,0,0
  0,55,45,46,0,0
  0,47,40,43,0,0
  0,55,61,54,1,0
  0,68,59,56,0,0
  0,31,36,46,0,0
  0,47,41,54,0,0
  0,63,59,57,0,0
  0,36,49,54,0,0
  0,68,59,71,0,0
  0,63,65,48,1,0
  0,55,41,40,0,0
  0,55,62,64,1,0
  0,52,41,51,0,0
  0,34,49,39,0,0
  0,50,31,40,0,0
  0,55,49,61,0,0
  0,52,62,66,1,0
  0,63,49,49,0,0
  1,68,62,65,1,65
  1,39,44,52,0,52
  1,44,44,46,0,46
  1,50,62,61,1,61
  1,71,65,72,1,72
  1,63,65,71,1,71
  1,34,44,40,0,40
\end{euleroutput}
\begin{eulerprompt}
>tok:=["female","read"]
\end{eulerprompt}
\begin{euleroutput}
  female
  read
\end{euleroutput}
\begin{eulerprompt}
>filename="mtcars.csv"; ...
>a = random(3,3), writematrix(a, filename);
\end{eulerprompt}
\begin{euleroutput}
       0.421434      0.128308      0.127171 
       0.679044      0.290959     0.0125867 
      0.0203728     0.0439462      0.404923 
\end{euleroutput}
\begin{eulerprompt}
>mean(a[1])
\end{eulerprompt}
\begin{euleroutput}
  0.225637223965
\end{euleroutput}
\begin{eulerprompt}
>a=random(3,3)
\end{eulerprompt}
\begin{euleroutput}
        0.21687       0.87946      0.727787 
       0.224946      0.514377      0.461146 
        0.48798       0.24778     0.0876318 
\end{euleroutput}
\eulerheading{Membaca dari Web}
\begin{eulercomment}
Situs web atau file dengan URL dapat dibuka di EMT dan dapat dibaca
baris demi baris.

Dalam contoh, kami membaca versi saat ini dari situs EMT. Kami
menggunakan ekspresi reguler untuk memindai "Versi ..." dalam sebuah
judul.
\end{eulercomment}
\begin{eulerprompt}
>function readversion() ...
\end{eulerprompt}
\begin{eulerudf}
  urlopen("http://www.euler-math-toolbox.de/Programs/Changes.html");
  repeat
    until urleof();
    s=urlgetline();
    k=strfind(s,"Version ",1);
    if k>0 then substring(s,k,strfind(s,"<",k)-1), break; endif;
  end;
  urlclose();
  endfunction
\end{eulerudf}
\begin{eulerprompt}
>readversion 
\end{eulerprompt}
\begin{euleroutput}
  Version 2022-05-18
\end{euleroutput}
\end{eulernotebook}
\end{document}
